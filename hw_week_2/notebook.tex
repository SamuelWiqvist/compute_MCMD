
% Default to the notebook output style

    


% Inherit from the specified cell style.




    
\documentclass[11pt]{article}

    
    
    \usepackage[T1]{fontenc}
    % Nicer default font (+ math font) than Computer Modern for most use cases
    \usepackage{mathpazo}

    % Basic figure setup, for now with no caption control since it's done
    % automatically by Pandoc (which extracts ![](path) syntax from Markdown).
    \usepackage{graphicx}
    % We will generate all images so they have a width \maxwidth. This means
    % that they will get their normal width if they fit onto the page, but
    % are scaled down if they would overflow the margins.
    \makeatletter
    \def\maxwidth{\ifdim\Gin@nat@width>\linewidth\linewidth
    \else\Gin@nat@width\fi}
    \makeatother
    \let\Oldincludegraphics\includegraphics
    % Set max figure width to be 80% of text width, for now hardcoded.
    \renewcommand{\includegraphics}[1]{\Oldincludegraphics[width=.8\maxwidth]{#1}}
    % Ensure that by default, figures have no caption (until we provide a
    % proper Figure object with a Caption API and a way to capture that
    % in the conversion process - todo).
    \usepackage{caption}
    \DeclareCaptionLabelFormat{nolabel}{}
    \captionsetup{labelformat=nolabel}

    \usepackage{adjustbox} % Used to constrain images to a maximum size 
    \usepackage{xcolor} % Allow colors to be defined
    \usepackage{enumerate} % Needed for markdown enumerations to work
    \usepackage{geometry} % Used to adjust the document margins
    \usepackage{amsmath} % Equations
    \usepackage{amssymb} % Equations
    \usepackage{textcomp} % defines textquotesingle
    % Hack from http://tex.stackexchange.com/a/47451/13684:
    \AtBeginDocument{%
        \def\PYZsq{\textquotesingle}% Upright quotes in Pygmentized code
    }
    \usepackage{upquote} % Upright quotes for verbatim code
    \usepackage{eurosym} % defines \euro
    \usepackage[mathletters]{ucs} % Extended unicode (utf-8) support
    \usepackage[utf8x]{inputenc} % Allow utf-8 characters in the tex document
    \usepackage{fancyvrb} % verbatim replacement that allows latex
    \usepackage{grffile} % extends the file name processing of package graphics 
                         % to support a larger range 
    % The hyperref package gives us a pdf with properly built
    % internal navigation ('pdf bookmarks' for the table of contents,
    % internal cross-reference links, web links for URLs, etc.)
    \usepackage{hyperref}
    \usepackage{longtable} % longtable support required by pandoc >1.10
    \usepackage{booktabs}  % table support for pandoc > 1.12.2
    \usepackage[inline]{enumitem} % IRkernel/repr support (it uses the enumerate* environment)
    \usepackage[normalem]{ulem} % ulem is needed to support strikethroughs (\sout)
                                % normalem makes italics be italics, not underlines
    

    
    
    % Colors for the hyperref package
    \definecolor{urlcolor}{rgb}{0,.145,.698}
    \definecolor{linkcolor}{rgb}{.71,0.21,0.01}
    \definecolor{citecolor}{rgb}{.12,.54,.11}

    % ANSI colors
    \definecolor{ansi-black}{HTML}{3E424D}
    \definecolor{ansi-black-intense}{HTML}{282C36}
    \definecolor{ansi-red}{HTML}{E75C58}
    \definecolor{ansi-red-intense}{HTML}{B22B31}
    \definecolor{ansi-green}{HTML}{00A250}
    \definecolor{ansi-green-intense}{HTML}{007427}
    \definecolor{ansi-yellow}{HTML}{DDB62B}
    \definecolor{ansi-yellow-intense}{HTML}{B27D12}
    \definecolor{ansi-blue}{HTML}{208FFB}
    \definecolor{ansi-blue-intense}{HTML}{0065CA}
    \definecolor{ansi-magenta}{HTML}{D160C4}
    \definecolor{ansi-magenta-intense}{HTML}{A03196}
    \definecolor{ansi-cyan}{HTML}{60C6C8}
    \definecolor{ansi-cyan-intense}{HTML}{258F8F}
    \definecolor{ansi-white}{HTML}{C5C1B4}
    \definecolor{ansi-white-intense}{HTML}{A1A6B2}

    % commands and environments needed by pandoc snippets
    % extracted from the output of `pandoc -s`
    \providecommand{\tightlist}{%
      \setlength{\itemsep}{0pt}\setlength{\parskip}{0pt}}
    \DefineVerbatimEnvironment{Highlighting}{Verbatim}{commandchars=\\\{\}}
    % Add ',fontsize=\small' for more characters per line
    \newenvironment{Shaded}{}{}
    \newcommand{\KeywordTok}[1]{\textcolor[rgb]{0.00,0.44,0.13}{\textbf{{#1}}}}
    \newcommand{\DataTypeTok}[1]{\textcolor[rgb]{0.56,0.13,0.00}{{#1}}}
    \newcommand{\DecValTok}[1]{\textcolor[rgb]{0.25,0.63,0.44}{{#1}}}
    \newcommand{\BaseNTok}[1]{\textcolor[rgb]{0.25,0.63,0.44}{{#1}}}
    \newcommand{\FloatTok}[1]{\textcolor[rgb]{0.25,0.63,0.44}{{#1}}}
    \newcommand{\CharTok}[1]{\textcolor[rgb]{0.25,0.44,0.63}{{#1}}}
    \newcommand{\StringTok}[1]{\textcolor[rgb]{0.25,0.44,0.63}{{#1}}}
    \newcommand{\CommentTok}[1]{\textcolor[rgb]{0.38,0.63,0.69}{\textit{{#1}}}}
    \newcommand{\OtherTok}[1]{\textcolor[rgb]{0.00,0.44,0.13}{{#1}}}
    \newcommand{\AlertTok}[1]{\textcolor[rgb]{1.00,0.00,0.00}{\textbf{{#1}}}}
    \newcommand{\FunctionTok}[1]{\textcolor[rgb]{0.02,0.16,0.49}{{#1}}}
    \newcommand{\RegionMarkerTok}[1]{{#1}}
    \newcommand{\ErrorTok}[1]{\textcolor[rgb]{1.00,0.00,0.00}{\textbf{{#1}}}}
    \newcommand{\NormalTok}[1]{{#1}}
    
    % Additional commands for more recent versions of Pandoc
    \newcommand{\ConstantTok}[1]{\textcolor[rgb]{0.53,0.00,0.00}{{#1}}}
    \newcommand{\SpecialCharTok}[1]{\textcolor[rgb]{0.25,0.44,0.63}{{#1}}}
    \newcommand{\VerbatimStringTok}[1]{\textcolor[rgb]{0.25,0.44,0.63}{{#1}}}
    \newcommand{\SpecialStringTok}[1]{\textcolor[rgb]{0.73,0.40,0.53}{{#1}}}
    \newcommand{\ImportTok}[1]{{#1}}
    \newcommand{\DocumentationTok}[1]{\textcolor[rgb]{0.73,0.13,0.13}{\textit{{#1}}}}
    \newcommand{\AnnotationTok}[1]{\textcolor[rgb]{0.38,0.63,0.69}{\textbf{\textit{{#1}}}}}
    \newcommand{\CommentVarTok}[1]{\textcolor[rgb]{0.38,0.63,0.69}{\textbf{\textit{{#1}}}}}
    \newcommand{\VariableTok}[1]{\textcolor[rgb]{0.10,0.09,0.49}{{#1}}}
    \newcommand{\ControlFlowTok}[1]{\textcolor[rgb]{0.00,0.44,0.13}{\textbf{{#1}}}}
    \newcommand{\OperatorTok}[1]{\textcolor[rgb]{0.40,0.40,0.40}{{#1}}}
    \newcommand{\BuiltInTok}[1]{{#1}}
    \newcommand{\ExtensionTok}[1]{{#1}}
    \newcommand{\PreprocessorTok}[1]{\textcolor[rgb]{0.74,0.48,0.00}{{#1}}}
    \newcommand{\AttributeTok}[1]{\textcolor[rgb]{0.49,0.56,0.16}{{#1}}}
    \newcommand{\InformationTok}[1]{\textcolor[rgb]{0.38,0.63,0.69}{\textbf{\textit{{#1}}}}}
    \newcommand{\WarningTok}[1]{\textcolor[rgb]{0.38,0.63,0.69}{\textbf{\textit{{#1}}}}}
    
    
    % Define a nice break command that doesn't care if a line doesn't already
    % exist.
    \def\br{\hspace*{\fill} \\* }
    % Math Jax compatability definitions
    \def\gt{>}
    \def\lt{<}
    % Document parameters
    \title{Steller population within clusters}
    
    
    

    % Pygments definitions
    
\makeatletter
\def\PY@reset{\let\PY@it=\relax \let\PY@bf=\relax%
    \let\PY@ul=\relax \let\PY@tc=\relax%
    \let\PY@bc=\relax \let\PY@ff=\relax}
\def\PY@tok#1{\csname PY@tok@#1\endcsname}
\def\PY@toks#1+{\ifx\relax#1\empty\else%
    \PY@tok{#1}\expandafter\PY@toks\fi}
\def\PY@do#1{\PY@bc{\PY@tc{\PY@ul{%
    \PY@it{\PY@bf{\PY@ff{#1}}}}}}}
\def\PY#1#2{\PY@reset\PY@toks#1+\relax+\PY@do{#2}}

\expandafter\def\csname PY@tok@w\endcsname{\def\PY@tc##1{\textcolor[rgb]{0.73,0.73,0.73}{##1}}}
\expandafter\def\csname PY@tok@c\endcsname{\let\PY@it=\textit\def\PY@tc##1{\textcolor[rgb]{0.25,0.50,0.50}{##1}}}
\expandafter\def\csname PY@tok@cp\endcsname{\def\PY@tc##1{\textcolor[rgb]{0.74,0.48,0.00}{##1}}}
\expandafter\def\csname PY@tok@k\endcsname{\let\PY@bf=\textbf\def\PY@tc##1{\textcolor[rgb]{0.00,0.50,0.00}{##1}}}
\expandafter\def\csname PY@tok@kp\endcsname{\def\PY@tc##1{\textcolor[rgb]{0.00,0.50,0.00}{##1}}}
\expandafter\def\csname PY@tok@kt\endcsname{\def\PY@tc##1{\textcolor[rgb]{0.69,0.00,0.25}{##1}}}
\expandafter\def\csname PY@tok@o\endcsname{\def\PY@tc##1{\textcolor[rgb]{0.40,0.40,0.40}{##1}}}
\expandafter\def\csname PY@tok@ow\endcsname{\let\PY@bf=\textbf\def\PY@tc##1{\textcolor[rgb]{0.67,0.13,1.00}{##1}}}
\expandafter\def\csname PY@tok@nb\endcsname{\def\PY@tc##1{\textcolor[rgb]{0.00,0.50,0.00}{##1}}}
\expandafter\def\csname PY@tok@nf\endcsname{\def\PY@tc##1{\textcolor[rgb]{0.00,0.00,1.00}{##1}}}
\expandafter\def\csname PY@tok@nc\endcsname{\let\PY@bf=\textbf\def\PY@tc##1{\textcolor[rgb]{0.00,0.00,1.00}{##1}}}
\expandafter\def\csname PY@tok@nn\endcsname{\let\PY@bf=\textbf\def\PY@tc##1{\textcolor[rgb]{0.00,0.00,1.00}{##1}}}
\expandafter\def\csname PY@tok@ne\endcsname{\let\PY@bf=\textbf\def\PY@tc##1{\textcolor[rgb]{0.82,0.25,0.23}{##1}}}
\expandafter\def\csname PY@tok@nv\endcsname{\def\PY@tc##1{\textcolor[rgb]{0.10,0.09,0.49}{##1}}}
\expandafter\def\csname PY@tok@no\endcsname{\def\PY@tc##1{\textcolor[rgb]{0.53,0.00,0.00}{##1}}}
\expandafter\def\csname PY@tok@nl\endcsname{\def\PY@tc##1{\textcolor[rgb]{0.63,0.63,0.00}{##1}}}
\expandafter\def\csname PY@tok@ni\endcsname{\let\PY@bf=\textbf\def\PY@tc##1{\textcolor[rgb]{0.60,0.60,0.60}{##1}}}
\expandafter\def\csname PY@tok@na\endcsname{\def\PY@tc##1{\textcolor[rgb]{0.49,0.56,0.16}{##1}}}
\expandafter\def\csname PY@tok@nt\endcsname{\let\PY@bf=\textbf\def\PY@tc##1{\textcolor[rgb]{0.00,0.50,0.00}{##1}}}
\expandafter\def\csname PY@tok@nd\endcsname{\def\PY@tc##1{\textcolor[rgb]{0.67,0.13,1.00}{##1}}}
\expandafter\def\csname PY@tok@s\endcsname{\def\PY@tc##1{\textcolor[rgb]{0.73,0.13,0.13}{##1}}}
\expandafter\def\csname PY@tok@sd\endcsname{\let\PY@it=\textit\def\PY@tc##1{\textcolor[rgb]{0.73,0.13,0.13}{##1}}}
\expandafter\def\csname PY@tok@si\endcsname{\let\PY@bf=\textbf\def\PY@tc##1{\textcolor[rgb]{0.73,0.40,0.53}{##1}}}
\expandafter\def\csname PY@tok@se\endcsname{\let\PY@bf=\textbf\def\PY@tc##1{\textcolor[rgb]{0.73,0.40,0.13}{##1}}}
\expandafter\def\csname PY@tok@sr\endcsname{\def\PY@tc##1{\textcolor[rgb]{0.73,0.40,0.53}{##1}}}
\expandafter\def\csname PY@tok@ss\endcsname{\def\PY@tc##1{\textcolor[rgb]{0.10,0.09,0.49}{##1}}}
\expandafter\def\csname PY@tok@sx\endcsname{\def\PY@tc##1{\textcolor[rgb]{0.00,0.50,0.00}{##1}}}
\expandafter\def\csname PY@tok@m\endcsname{\def\PY@tc##1{\textcolor[rgb]{0.40,0.40,0.40}{##1}}}
\expandafter\def\csname PY@tok@gh\endcsname{\let\PY@bf=\textbf\def\PY@tc##1{\textcolor[rgb]{0.00,0.00,0.50}{##1}}}
\expandafter\def\csname PY@tok@gu\endcsname{\let\PY@bf=\textbf\def\PY@tc##1{\textcolor[rgb]{0.50,0.00,0.50}{##1}}}
\expandafter\def\csname PY@tok@gd\endcsname{\def\PY@tc##1{\textcolor[rgb]{0.63,0.00,0.00}{##1}}}
\expandafter\def\csname PY@tok@gi\endcsname{\def\PY@tc##1{\textcolor[rgb]{0.00,0.63,0.00}{##1}}}
\expandafter\def\csname PY@tok@gr\endcsname{\def\PY@tc##1{\textcolor[rgb]{1.00,0.00,0.00}{##1}}}
\expandafter\def\csname PY@tok@ge\endcsname{\let\PY@it=\textit}
\expandafter\def\csname PY@tok@gs\endcsname{\let\PY@bf=\textbf}
\expandafter\def\csname PY@tok@gp\endcsname{\let\PY@bf=\textbf\def\PY@tc##1{\textcolor[rgb]{0.00,0.00,0.50}{##1}}}
\expandafter\def\csname PY@tok@go\endcsname{\def\PY@tc##1{\textcolor[rgb]{0.53,0.53,0.53}{##1}}}
\expandafter\def\csname PY@tok@gt\endcsname{\def\PY@tc##1{\textcolor[rgb]{0.00,0.27,0.87}{##1}}}
\expandafter\def\csname PY@tok@err\endcsname{\def\PY@bc##1{\setlength{\fboxsep}{0pt}\fcolorbox[rgb]{1.00,0.00,0.00}{1,1,1}{\strut ##1}}}
\expandafter\def\csname PY@tok@kc\endcsname{\let\PY@bf=\textbf\def\PY@tc##1{\textcolor[rgb]{0.00,0.50,0.00}{##1}}}
\expandafter\def\csname PY@tok@kd\endcsname{\let\PY@bf=\textbf\def\PY@tc##1{\textcolor[rgb]{0.00,0.50,0.00}{##1}}}
\expandafter\def\csname PY@tok@kn\endcsname{\let\PY@bf=\textbf\def\PY@tc##1{\textcolor[rgb]{0.00,0.50,0.00}{##1}}}
\expandafter\def\csname PY@tok@kr\endcsname{\let\PY@bf=\textbf\def\PY@tc##1{\textcolor[rgb]{0.00,0.50,0.00}{##1}}}
\expandafter\def\csname PY@tok@bp\endcsname{\def\PY@tc##1{\textcolor[rgb]{0.00,0.50,0.00}{##1}}}
\expandafter\def\csname PY@tok@fm\endcsname{\def\PY@tc##1{\textcolor[rgb]{0.00,0.00,1.00}{##1}}}
\expandafter\def\csname PY@tok@vc\endcsname{\def\PY@tc##1{\textcolor[rgb]{0.10,0.09,0.49}{##1}}}
\expandafter\def\csname PY@tok@vg\endcsname{\def\PY@tc##1{\textcolor[rgb]{0.10,0.09,0.49}{##1}}}
\expandafter\def\csname PY@tok@vi\endcsname{\def\PY@tc##1{\textcolor[rgb]{0.10,0.09,0.49}{##1}}}
\expandafter\def\csname PY@tok@vm\endcsname{\def\PY@tc##1{\textcolor[rgb]{0.10,0.09,0.49}{##1}}}
\expandafter\def\csname PY@tok@sa\endcsname{\def\PY@tc##1{\textcolor[rgb]{0.73,0.13,0.13}{##1}}}
\expandafter\def\csname PY@tok@sb\endcsname{\def\PY@tc##1{\textcolor[rgb]{0.73,0.13,0.13}{##1}}}
\expandafter\def\csname PY@tok@sc\endcsname{\def\PY@tc##1{\textcolor[rgb]{0.73,0.13,0.13}{##1}}}
\expandafter\def\csname PY@tok@dl\endcsname{\def\PY@tc##1{\textcolor[rgb]{0.73,0.13,0.13}{##1}}}
\expandafter\def\csname PY@tok@s2\endcsname{\def\PY@tc##1{\textcolor[rgb]{0.73,0.13,0.13}{##1}}}
\expandafter\def\csname PY@tok@sh\endcsname{\def\PY@tc##1{\textcolor[rgb]{0.73,0.13,0.13}{##1}}}
\expandafter\def\csname PY@tok@s1\endcsname{\def\PY@tc##1{\textcolor[rgb]{0.73,0.13,0.13}{##1}}}
\expandafter\def\csname PY@tok@mb\endcsname{\def\PY@tc##1{\textcolor[rgb]{0.40,0.40,0.40}{##1}}}
\expandafter\def\csname PY@tok@mf\endcsname{\def\PY@tc##1{\textcolor[rgb]{0.40,0.40,0.40}{##1}}}
\expandafter\def\csname PY@tok@mh\endcsname{\def\PY@tc##1{\textcolor[rgb]{0.40,0.40,0.40}{##1}}}
\expandafter\def\csname PY@tok@mi\endcsname{\def\PY@tc##1{\textcolor[rgb]{0.40,0.40,0.40}{##1}}}
\expandafter\def\csname PY@tok@il\endcsname{\def\PY@tc##1{\textcolor[rgb]{0.40,0.40,0.40}{##1}}}
\expandafter\def\csname PY@tok@mo\endcsname{\def\PY@tc##1{\textcolor[rgb]{0.40,0.40,0.40}{##1}}}
\expandafter\def\csname PY@tok@ch\endcsname{\let\PY@it=\textit\def\PY@tc##1{\textcolor[rgb]{0.25,0.50,0.50}{##1}}}
\expandafter\def\csname PY@tok@cm\endcsname{\let\PY@it=\textit\def\PY@tc##1{\textcolor[rgb]{0.25,0.50,0.50}{##1}}}
\expandafter\def\csname PY@tok@cpf\endcsname{\let\PY@it=\textit\def\PY@tc##1{\textcolor[rgb]{0.25,0.50,0.50}{##1}}}
\expandafter\def\csname PY@tok@c1\endcsname{\let\PY@it=\textit\def\PY@tc##1{\textcolor[rgb]{0.25,0.50,0.50}{##1}}}
\expandafter\def\csname PY@tok@cs\endcsname{\let\PY@it=\textit\def\PY@tc##1{\textcolor[rgb]{0.25,0.50,0.50}{##1}}}

\def\PYZbs{\char`\\}
\def\PYZus{\char`\_}
\def\PYZob{\char`\{}
\def\PYZcb{\char`\}}
\def\PYZca{\char`\^}
\def\PYZam{\char`\&}
\def\PYZlt{\char`\<}
\def\PYZgt{\char`\>}
\def\PYZsh{\char`\#}
\def\PYZpc{\char`\%}
\def\PYZdl{\char`\$}
\def\PYZhy{\char`\-}
\def\PYZsq{\char`\'}
\def\PYZdq{\char`\"}
\def\PYZti{\char`\~}
% for compatibility with earlier versions
\def\PYZat{@}
\def\PYZlb{[}
\def\PYZrb{]}
\makeatother


    % Exact colors from NB
    \definecolor{incolor}{rgb}{0.0, 0.0, 0.5}
    \definecolor{outcolor}{rgb}{0.545, 0.0, 0.0}



    
    % Prevent overflowing lines due to hard-to-break entities
    \sloppy 
    % Setup hyperref package
    \hypersetup{
      breaklinks=true,  % so long urls are correctly broken across lines
      colorlinks=true,
      urlcolor=urlcolor,
      linkcolor=linkcolor,
      citecolor=citecolor,
      }
    % Slightly bigger margins than the latex defaults
    
    \geometry{verbose,tmargin=1in,bmargin=1in,lmargin=1in,rmargin=1in}
    
    

    \begin{document}
    
    
    \maketitle
    
    

    
    \textbf{COMPUT MCMD course sping 2019 home work for week 2}

\textbf{By: Samuel Wiqvist}

    \subsubsection{Introduction}\label{introduction}

    We will use Monte Carlo methods to analyses steller population within
clusters. In our calculations we set \(M_{\circ} = 1\).

    \paragraph{Set-up}\label{set-up}

    The first step it to load the packages that we will use for our
calculations.

    \begin{Verbatim}[commandchars=\\\{\}]
{\color{incolor}In [{\color{incolor}15}]:} \PY{k}{using} \PY{n}{PyPlot} \PY{c}{\PYZsh{} for plotting}
         \PY{k}{using} \PY{n}{Statistics} \PY{c}{\PYZsh{} for compution statistics}
\end{Verbatim}


    \paragraph{The mass distribution for
stars}\label{the-mass-distribution-for-stars}

    We first plot the distribution (i.e the power law) for masses in stars.

    \begin{Verbatim}[commandchars=\\\{\}]
{\color{incolor}In [{\color{incolor}6}]:} \PY{c}{\PYZsh{} density function for the mass }
        \PY{k}{function} \PY{n}{f}\PY{p}{(}\PY{n}{x}\PY{p}{)}
        
            \PY{k}{if} \PY{n}{x} \PY{o}{\PYZgt{}=} \PY{l+m+mf}{0.08} \PY{o}{\PYZam{}\PYZam{}} \PY{n}{x} \PY{o}{\PYZlt{}=} \PY{l+m+mf}{0.5}
                \PY{n+nb}{γ} \PY{o}{=} \PY{l+m+mf}{1.3}
            \PY{k}{elseif} \PY{n}{x} \PY{o}{\PYZgt{}=} \PY{l+m+mf}{0.5} \PY{o}{\PYZam{}\PYZam{}} \PY{n}{x} \PY{o}{\PYZlt{}=} \PY{l+m+mi}{1}
                \PY{n+nb}{γ} \PY{o}{=} \PY{l+m+mf}{2.2}
            \PY{k}{else}
                \PY{n+nb}{γ} \PY{o}{=} \PY{l+m+mf}{2.7}
            \PY{k}{end}
        
            \PY{k}{return} \PY{n}{x}\PY{o}{\PYZca{}}\PY{p}{(}\PY{o}{\PYZhy{}}\PY{n+nb}{γ}\PY{p}{)}
        
        \PY{k}{end}
        
        \PY{c}{\PYZsh{} compute }
        \PY{n}{x} \PY{o}{=} \PY{n}{LinRange}\PY{p}{(}\PY{l+m+mf}{0.08}\PY{p}{,}\PY{l+m+mi}{120}\PY{p}{,} \PY{l+m+mi}{200}\PY{p}{)}
        \PY{n}{density} \PY{o}{=} \PY{n}{zeros}\PY{p}{(}\PY{n}{length}\PY{p}{(}\PY{n}{x}\PY{p}{)}\PY{p}{)}
        \PY{k}{for} \PY{n}{i} \PY{k+kp}{in} \PY{l+m+mi}{1}\PY{o}{:}\PY{n}{length}\PY{p}{(}\PY{n}{x}\PY{p}{)}\PY{p}{;} \PY{n}{density}\PY{p}{[}\PY{n}{i}\PY{p}{]} \PY{o}{=} \PY{n}{f}\PY{p}{(}\PY{n}{x}\PY{p}{[}\PY{n}{i}\PY{p}{]}\PY{p}{)}\PY{p}{;} \PY{k}{end}
        
        \PY{c}{\PYZsh{} plot desity function}
        \PY{n}{PyPlot}\PY{o}{.}\PY{n}{figure}\PY{p}{(}\PY{p}{)}
        \PY{n}{PyPlot}\PY{o}{.}\PY{n}{plot}\PY{p}{(}\PY{n}{x}\PY{p}{,} \PY{n}{density}\PY{p}{)}
        \PY{n}{PyPlot}\PY{o}{.}\PY{n}{xlabel}\PY{p}{(}\PY{l+s}{\PYZdq{}}\PY{l+s}{M}\PY{l+s}{a}\PY{l+s}{s}\PY{l+s}{s}\PY{l+s}{\PYZdq{}}\PY{p}{)}
        \PY{n}{PyPlot}\PY{o}{.}\PY{n}{ylabel}\PY{p}{(}\PY{l+s}{\PYZdq{}}\PY{l+s}{D}\PY{l+s}{e}\PY{l+s}{n}\PY{l+s}{s}\PY{l+s}{i}\PY{l+s}{t}\PY{l+s}{y}\PY{l+s}{\PYZdq{}}\PY{p}{)}\PY{p}{;}
\end{Verbatim}


    \begin{center}
    \adjustimage{max size={0.9\linewidth}{0.9\paperheight}}{output_8_0.png}
    \end{center}
    { \hspace*{\fill} \\}
    
    \paragraph{Sampling stars with masses accoring to the density
function}\label{sampling-stars-with-masses-accoring-to-the-density-function}

    To sample stars with masses accorsing to the density funciton that we
have been given we use the hit-and-miss method.

    \begin{Verbatim}[commandchars=\\\{\}]
{\color{incolor}In [{\color{incolor} }]:} \PY{c}{\PYZsh{} hit\PYZhy{}and\PYZhy{}miss method to sample stars with masses accoring to the density function f}
        \PY{k}{function} \PY{n}{hitandmiss}\PY{p}{(}\PY{n}{N\PYZus{}stars}\PY{p}{)}
        
            \PY{n}{y\PYZus{}max} \PY{o}{=} \PY{n}{round}\PY{p}{(}\PY{n}{f}\PY{p}{(}\PY{l+m+mf}{0.08}\PY{p}{)}\PY{p}{)} 
        
            \PY{n}{N\PYZus{}acc} \PY{o}{=} \PY{l+m+mi}{0}
        
            \PY{n}{stars} \PY{o}{=} \PY{n}{zeros}\PY{p}{(}\PY{n}{N\PYZus{}stars}\PY{p}{)}
        
            \PY{k}{for} \PY{n}{i} \PY{o}{=} \PY{l+m+mi}{1}\PY{o}{:}\PY{n}{N\PYZus{}stars}
        
                \PY{n}{generate} \PY{o}{=} \PY{k+kc}{true}
        
                \PY{k}{while} \PY{n}{generate}
                    \PY{n}{x\PYZus{}star} \PY{o}{=} \PY{l+m+mf}{0.08}\PY{o}{+}\PY{p}{(}\PY{l+m+mi}{120}\PY{o}{\PYZhy{}}\PY{l+m+mf}{0.08}\PY{p}{)}\PY{o}{*}\PY{n}{rand}\PY{p}{(}\PY{p}{)}
                    \PY{n}{y\PYZus{}star} \PY{o}{=} \PY{n}{y\PYZus{}max}\PY{o}{*}\PY{n}{rand}\PY{p}{(}\PY{p}{)}
        
                    \PY{k}{if} \PY{n}{y\PYZus{}star} \PY{o}{\PYZlt{}=} \PY{n}{f}\PY{p}{(}\PY{n}{x\PYZus{}star}\PY{p}{)}
                        \PY{n}{generate} \PY{o}{=} \PY{k+kc}{false}
                        \PY{n}{stars}\PY{p}{[}\PY{n}{i}\PY{p}{]} \PY{o}{=} \PY{n}{x\PYZus{}star}
                    \PY{k}{end}
        
                \PY{k}{end}
        
            \PY{k}{end}
        
            \PY{k}{return} \PY{n}{stars}
        
        \PY{k}{end}
\end{Verbatim}


    \paragraph{The probability for a
supernova}\label{the-probability-for-a-supernova}

    If the cluset contains stars with masses larger than \(8 M_{\circ}\),
then these stars will explode as suvernovae. We will now compute the
probability that a cluster with \(N_{\star}=100,300,1000\) stars
contains contain at least one supernove.

We now compute the probabilities by sampling clusters with
\(N_{\star}=100,300,1000\) stars our hit-and-miss method.

    \begin{Verbatim}[commandchars=\\\{\}]
{\color{incolor}In [{\color{incolor}8}]:} \PY{c}{\PYZsh{} set number of starts to generate}
        \PY{n}{N\PYZus{}stars} \PY{o}{=} \PY{p}{[}\PY{l+m+mi}{100}\PY{p}{,}\PY{l+m+mi}{300}\PY{p}{,}\PY{l+m+mi}{1000}\PY{p}{]}
        \PY{n}{prob\PYZus{}supernova} \PY{o}{=} \PY{n}{zeros}\PY{p}{(}\PY{n}{length}\PY{p}{(}\PY{n}{N\PYZus{}stars}\PY{p}{)}\PY{p}{)}
        
        \PY{c}{\PYZsh{} genreate clusters and compute probabilityis }
        \PY{k}{for} \PY{n}{i} \PY{k+kp}{in} \PY{l+m+mi}{1}\PY{o}{:}\PY{n}{length}\PY{p}{(}\PY{n}{N\PYZus{}stars}\PY{p}{)}
            \PY{n}{prob\PYZus{}supernova}\PY{p}{[}\PY{n}{i}\PY{p}{]} \PY{o}{=} \PY{n}{length}\PY{p}{(}\PY{n}{findall}\PY{p}{(}\PY{n}{x} \PY{o}{\PYZhy{}}\PY{o}{\PYZgt{}} \PY{n}{x} \PY{o}{\PYZgt{}} \PY{l+m+mi}{8}\PY{p}{,} \PY{n}{hitandmiss}\PY{p}{(}\PY{n}{N\PYZus{}stars}\PY{p}{[}\PY{n}{i}\PY{p}{]}\PY{p}{)}\PY{p}{)}\PY{p}{)}\PY{o}{/}\PY{n}{N\PYZus{}stars}\PY{p}{[}\PY{n}{i}\PY{p}{]}
        \PY{k}{end}
        
        \PY{c}{\PYZsh{} print computed prob for a supernovae}
        \PY{n}{print}\PY{p}{(}\PY{n}{prob\PYZus{}supernova}\PY{p}{)}
\end{Verbatim}


    \begin{Verbatim}[commandchars=\\\{\}]
[0.0, 0.00666667, 0.007]
    \end{Verbatim}

    We can now conclude that we obtain following probabilityies for a
supernovae:

\(P(\text{at least on star with mass} > M_{\circ})_{N_{\star}=100} = 0\)

\(P(\text{at least on star with mass} > M_{\circ})_{N_{\star}=300} = 0.0067\)

\(P(\text{at least on star with mass} > M_{\circ})_{N_{\star}=1000} = 0.007\)

    The next task is to probability for a supernove for
\(50 \leq N_{\star} \leq 1000\).

    \begin{Verbatim}[commandchars=\\\{\}]
{\color{incolor}In [{\color{incolor}16}]:} \PY{c}{\PYZsh{} set number of starts to generate}
         \PY{n}{N\PYZus{}stars} \PY{o}{=} \PY{n}{floor}\PY{o}{.}\PY{p}{(}\PY{k+kt}{Int}\PY{p}{,}\PY{n}{LinRange}\PY{p}{(}\PY{l+m+mi}{50}\PY{p}{,}\PY{l+m+mi}{5000}\PY{p}{,}\PY{l+m+mi}{200}\PY{p}{)}\PY{p}{)}
         \PY{n}{prob\PYZus{}supernova} \PY{o}{=} \PY{n}{zeros}\PY{p}{(}\PY{n}{length}\PY{p}{(}\PY{n}{N\PYZus{}stars}\PY{p}{)}\PY{p}{)}
         
         \PY{c}{\PYZsh{} genreate clusters and compute probabilityis }
         \PY{k}{for} \PY{n}{i} \PY{k+kp}{in} \PY{l+m+mi}{1}\PY{o}{:}\PY{n}{length}\PY{p}{(}\PY{n}{N\PYZus{}stars}\PY{p}{)}
             \PY{n}{prob\PYZus{}supernova}\PY{p}{[}\PY{n}{i}\PY{p}{]} \PY{o}{=} \PY{n}{length}\PY{p}{(}\PY{n}{findall}\PY{p}{(}\PY{n}{x} \PY{o}{\PYZhy{}}\PY{o}{\PYZgt{}} \PY{n}{x} \PY{o}{\PYZgt{}} \PY{l+m+mi}{8}\PY{p}{,} \PY{n}{hitandmiss}\PY{p}{(}\PY{n}{N\PYZus{}stars}\PY{p}{[}\PY{n}{i}\PY{p}{]}\PY{p}{)}\PY{p}{)}\PY{p}{)}\PY{o}{/}\PY{n}{N\PYZus{}stars}\PY{p}{[}\PY{n}{i}\PY{p}{]}
         \PY{k}{end}
         
         \PY{c}{\PYZsh{} plot prob supernova}
         \PY{n}{PyPlot}\PY{o}{.}\PY{n}{figure}\PY{p}{(}\PY{p}{)}
         \PY{n}{PyPlot}\PY{o}{.}\PY{n}{plot}\PY{p}{(}\PY{n}{N\PYZus{}stars}\PY{p}{,} \PY{n}{prob\PYZus{}supernova}\PY{p}{)}
         \PY{n}{PyPlot}\PY{o}{.}\PY{n}{xlabel}\PY{p}{(}\PY{l+s}{\PYZdq{}}\PY{l+s}{N}\PY{l+s}{u}\PY{l+s}{m}\PY{l+s}{b}\PY{l+s}{e}\PY{l+s}{r}\PY{l+s}{ }\PY{l+s}{o}\PY{l+s}{f}\PY{l+s}{ }\PY{l+s}{s}\PY{l+s}{t}\PY{l+s}{a}\PY{l+s}{r}\PY{l+s}{s}\PY{l+s}{ }\PY{l+s}{i}\PY{l+s}{n}\PY{l+s}{ }\PY{l+s}{c}\PY{l+s}{l}\PY{l+s}{u}\PY{l+s}{s}\PY{l+s}{e}\PY{l+s}{r}\PY{l+s}{\PYZdq{}}\PY{p}{)}
         \PY{n}{PyPlot}\PY{o}{.}\PY{n}{ylabel}\PY{p}{(}\PY{l+s}{\PYZdq{}}\PY{l+s}{E}\PY{l+s}{s}\PY{l+s}{t}\PY{l+s}{.}\PY{l+s}{ }\PY{l+s}{p}\PY{l+s}{r}\PY{l+s}{o}\PY{l+s}{b}\PY{l+s}{.}\PY{l+s}{ }\PY{l+s}{f}\PY{l+s}{o}\PY{l+s}{r}\PY{l+s}{ }\PY{l+s}{a}\PY{l+s}{ }\PY{l+s}{s}\PY{l+s}{u}\PY{l+s}{p}\PY{l+s}{e}\PY{l+s}{r}\PY{l+s}{n}\PY{l+s}{o}\PY{l+s}{v}\PY{l+s}{a}\PY{l+s}{\PYZdq{}}\PY{p}{)}
\end{Verbatim}


    \begin{center}
    \adjustimage{max size={0.9\linewidth}{0.9\paperheight}}{output_17_0.png}
    \end{center}
    { \hspace*{\fill} \\}
    
\begin{Verbatim}[commandchars=\\\{\}]
{\color{outcolor}Out[{\color{outcolor}16}]:} PyObject Text(24,0.5,'Est. prob. for a supernova')
\end{Verbatim}
            
    We can now conclude that the probability for a supernova is low if the
cluster contains few stars \(N_{\star} < 1000\), and the the probability
for at least on supernova converges to approximatly \(0.004\).

    The next task is to compute the mean, median and lower qunatile (the Q1
quantile) expected number of supernovas for a cluster with
\(N_{\star} = 5000\) stars. For this task we genreate 100 clusters with
\(N_{\star} = 5000\) stars and compute statistics for the population of
clusters.

    \begin{Verbatim}[commandchars=\\\{\}]
{\color{incolor}In [{\color{incolor}18}]:} \PY{c}{\PYZsh{} set number of starts and number of clusters to generate}
         \PY{n}{N\PYZus{}clusters} \PY{o}{=} \PY{l+m+mi}{100}
         \PY{n}{N\PYZus{}stars} \PY{o}{=} \PY{l+m+mi}{5000}
         \PY{n}{prob\PYZus{}supernova} \PY{o}{=} \PY{n}{zeros}\PY{p}{(}\PY{n}{N\PYZus{}clusters}\PY{p}{)}
         
         \PY{c}{\PYZsh{} genreate clusters and compute probabilityis }
         \PY{k}{for} \PY{n}{i} \PY{k+kp}{in} \PY{l+m+mi}{1}\PY{o}{:}\PY{n}{N\PYZus{}clusters}
             \PY{n}{prob\PYZus{}supernova}\PY{p}{[}\PY{n}{i}\PY{p}{]} \PY{o}{=} \PY{n}{length}\PY{p}{(}\PY{n}{findall}\PY{p}{(}\PY{n}{x} \PY{o}{\PYZhy{}}\PY{o}{\PYZgt{}} \PY{n}{x} \PY{o}{\PYZgt{}} \PY{l+m+mi}{8}\PY{p}{,} \PY{n}{hitandmiss}\PY{p}{(}\PY{n}{N\PYZus{}stars}\PY{p}{)}\PY{p}{)}\PY{p}{)}\PY{o}{/}\PY{n}{N\PYZus{}stars}
         \PY{k}{end}
         
         \PY{c}{\PYZsh{} print statistics for the population of clusters }
         \PY{n}{println}\PY{p}{(}\PY{n}{mean}\PY{p}{(}\PY{n}{prob\PYZus{}supernova}\PY{p}{)}\PY{p}{)}
         \PY{n}{println}\PY{p}{(}\PY{n}{median}\PY{p}{(}\PY{n}{prob\PYZus{}supernova}\PY{p}{)}\PY{p}{)}
         \PY{n}{println}\PY{p}{(}\PY{n}{quantile}\PY{p}{(}\PY{n}{prob\PYZus{}supernova}\PY{p}{,} \PY{l+m+mf}{0.25}\PY{p}{)}\PY{p}{)}
\end{Verbatim}


    \begin{Verbatim}[commandchars=\\\{\}]
0.0036420000000000003
0.0036
0.003

    \end{Verbatim}

    We obtain following statistics:

\(\text{mean}(P(\text{at least on star with mass} > M_{\circ})_{N_{\star}=5000}) = 0.0036\)

\(\text{median}(P(\text{at least on star with mass} > M_{\circ})_{N_{\star}=5000}) = 0.0036\)

\(Q1(P(\text{at least on star with mass} > M_{\circ})_{N_{\star}=5000})) = 0.03\)

    We conclude that the average probabiliy for supernova in a cluster with
\(N_{\star} = 5000\) stars is \(0.0036\), which is in line with results
we obtained earlier.

    \paragraph{\texorpdfstring{Number of stars in a cluster with a star with
\(M_{\text{sn}} \geq 25 M_{\circ}\)}{Number of stars in a cluster with a star with M\_\{\textbackslash{}text\{sn\}\} \textbackslash{}geq 25 M\_\{\textbackslash{}circ\}}}\label{number-of-stars-in-a-cluster-with-a-star-with-m_textsn-geq-25-m_circ}

    We are now interested in inverstingating how many stars there are in a
cluster such that the cluster contains a supernova with mass
\(M_{\text{sn}} \geq 25 M_{\circ}\). This is question is of particular
interest since our own sum is beleved to have formed in such a cluster.

To investigate this we compaute the probability for having a supernova
with \(M_{\text{sn}} \geq 25 M_{\circ}\) in clusters with different
number of stars \(N_{\star}\).

    \begin{Verbatim}[commandchars=\\\{\}]
{\color{incolor}In [{\color{incolor}26}]:} \PY{c}{\PYZsh{} set number of starts to generate}
         \PY{n}{N\PYZus{}stars} \PY{o}{=} \PY{n}{floor}\PY{o}{.}\PY{p}{(}\PY{k+kt}{Int}\PY{p}{,}\PY{n}{LinRange}\PY{p}{(}\PY{l+m+mi}{100}\PY{p}{,}\PY{l+m+mi}{10}\PY{o}{\PYZca{}}\PY{l+m+mi}{4}\PY{p}{,}\PY{l+m+mi}{100}\PY{p}{)}\PY{p}{)}
         \PY{n}{prob\PYZus{}supernova} \PY{o}{=} \PY{n}{zeros}\PY{p}{(}\PY{n}{length}\PY{p}{(}\PY{n}{N\PYZus{}stars}\PY{p}{)}\PY{p}{)}
         
         \PY{c}{\PYZsh{} genreate clusters and compute probabilityis }
         \PY{k}{for} \PY{n}{i} \PY{k+kp}{in} \PY{l+m+mi}{1}\PY{o}{:}\PY{n}{length}\PY{p}{(}\PY{n}{N\PYZus{}stars}\PY{p}{)}
             \PY{n}{prob\PYZus{}supernova}\PY{p}{[}\PY{n}{i}\PY{p}{]} \PY{o}{=} \PY{n}{length}\PY{p}{(}\PY{n}{findall}\PY{p}{(}\PY{n}{x} \PY{o}{\PYZhy{}}\PY{o}{\PYZgt{}} \PY{n}{x} \PY{o}{\PYZgt{}} \PY{l+m+mi}{25}\PY{p}{,} \PY{n}{hitandmiss}\PY{p}{(}\PY{n}{N\PYZus{}stars}\PY{p}{[}\PY{n}{i}\PY{p}{]}\PY{p}{)}\PY{p}{)}\PY{p}{)}\PY{o}{/}\PY{n}{N\PYZus{}stars}\PY{p}{[}\PY{n}{i}\PY{p}{]}
         \PY{k}{end}
         
         \PY{c}{\PYZsh{} plot prob for supernova}
         \PY{n}{PyPlot}\PY{o}{.}\PY{n}{figure}\PY{p}{(}\PY{p}{)}
         \PY{n}{PyPlot}\PY{o}{.}\PY{n}{plot}\PY{p}{(}\PY{n}{N\PYZus{}stars}\PY{p}{,} \PY{n}{prob\PYZus{}supernova}\PY{p}{)}
         \PY{n}{PyPlot}\PY{o}{.}\PY{n}{plot}\PY{p}{(}\PY{n}{N\PYZus{}stars}\PY{p}{,} \PY{n}{zeros}\PY{p}{(}\PY{n}{length}\PY{p}{(}\PY{n}{N\PYZus{}stars}\PY{p}{)}\PY{p}{)}\PY{p}{,} \PY{l+s}{\PYZdq{}}\PY{l+s}{k}\PY{l+s}{\PYZdq{}}\PY{p}{)}
         \PY{n}{PyPlot}\PY{o}{.}\PY{n}{xlabel}\PY{p}{(}\PY{l+s}{\PYZdq{}}\PY{l+s}{N}\PY{l+s}{u}\PY{l+s}{m}\PY{l+s}{b}\PY{l+s}{e}\PY{l+s}{r}\PY{l+s}{ }\PY{l+s}{o}\PY{l+s}{f}\PY{l+s}{ }\PY{l+s}{s}\PY{l+s}{t}\PY{l+s}{a}\PY{l+s}{r}\PY{l+s}{s}\PY{l+s}{ }\PY{l+s}{i}\PY{l+s}{n}\PY{l+s}{ }\PY{l+s}{c}\PY{l+s}{l}\PY{l+s}{u}\PY{l+s}{s}\PY{l+s}{e}\PY{l+s}{r}\PY{l+s}{\PYZdq{}}\PY{p}{)}
         \PY{n}{PyPlot}\PY{o}{.}\PY{n}{ylabel}\PY{p}{(}\PY{l+s}{\PYZdq{}}\PY{l+s}{E}\PY{l+s}{s}\PY{l+s}{t}\PY{l+s}{.}\PY{l+s}{ }\PY{l+s}{p}\PY{l+s}{r}\PY{l+s}{o}\PY{l+s}{b}\PY{l+s}{.}\PY{l+s}{ }\PY{l+s}{f}\PY{l+s}{o}\PY{l+s}{r}\PY{l+s}{ }\PY{l+s}{a}\PY{l+s}{ }\PY{l+s}{s}\PY{l+s}{u}\PY{l+s}{p}\PY{l+s}{e}\PY{l+s}{r}\PY{l+s}{n}\PY{l+s}{o}\PY{l+s}{v}\PY{l+s}{a}\PY{l+s}{ }\PY{l+s}{w}\PY{l+s}{i}\PY{l+s}{t}\PY{l+s}{h}\PY{l+s}{ }\PY{l+s}{l}\PY{l+s}{a}\PY{l+s}{r}\PY{l+s}{g}\PY{l+s}{e}\PY{l+s}{ }\PY{l+s}{m}\PY{l+s}{a}\PY{l+s}{s}\PY{l+s}{s}\PY{l+s}{\PYZdq{}}\PY{p}{)}
\end{Verbatim}


    \begin{center}
    \adjustimage{max size={0.9\linewidth}{0.9\paperheight}}{output_25_0.png}
    \end{center}
    { \hspace*{\fill} \\}
    
\begin{Verbatim}[commandchars=\\\{\}]
{\color{outcolor}Out[{\color{outcolor}26}]:} PyObject Text(24,0.5,'Est. prob. for a supernova with large mass')
\end{Verbatim}
            
    Out simulations show that a cluster that contains a supernova with
\(M_{\text{sn}} \geq 25 M_{\circ}\) likely containes more than
\(N_{\star} = 4000\) stars, since probability for a supernova with
\(M_{\text{sn}} \geq 25 M_{\circ}\) is zeros for some clusters with less
then \(N_{\star} = 4000\) stars.


    % Add a bibliography block to the postdoc
    
    
    
    \end{document}
