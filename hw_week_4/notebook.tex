
% Default to the notebook output style

    


% Inherit from the specified cell style.




    
\documentclass[11pt]{article}

    
    
    \usepackage[T1]{fontenc}
    % Nicer default font (+ math font) than Computer Modern for most use cases
    \usepackage{mathpazo}

    % Basic figure setup, for now with no caption control since it's done
    % automatically by Pandoc (which extracts ![](path) syntax from Markdown).
    \usepackage{graphicx}
    % We will generate all images so they have a width \maxwidth. This means
    % that they will get their normal width if they fit onto the page, but
    % are scaled down if they would overflow the margins.
    \makeatletter
    \def\maxwidth{\ifdim\Gin@nat@width>\linewidth\linewidth
    \else\Gin@nat@width\fi}
    \makeatother
    \let\Oldincludegraphics\includegraphics
    % Set max figure width to be 80% of text width, for now hardcoded.
    \renewcommand{\includegraphics}[1]{\Oldincludegraphics[width=.8\maxwidth]{#1}}
    % Ensure that by default, figures have no caption (until we provide a
    % proper Figure object with a Caption API and a way to capture that
    % in the conversion process - todo).
    \usepackage{caption}
    \DeclareCaptionLabelFormat{nolabel}{}
    \captionsetup{labelformat=nolabel}

    \usepackage{adjustbox} % Used to constrain images to a maximum size 
    \usepackage{xcolor} % Allow colors to be defined
    \usepackage{enumerate} % Needed for markdown enumerations to work
    \usepackage{geometry} % Used to adjust the document margins
    \usepackage{amsmath} % Equations
    \usepackage{amssymb} % Equations
    \usepackage{textcomp} % defines textquotesingle
    % Hack from http://tex.stackexchange.com/a/47451/13684:
    \AtBeginDocument{%
        \def\PYZsq{\textquotesingle}% Upright quotes in Pygmentized code
    }
    \usepackage{upquote} % Upright quotes for verbatim code
    \usepackage{eurosym} % defines \euro
    \usepackage[mathletters]{ucs} % Extended unicode (utf-8) support
    \usepackage[utf8x]{inputenc} % Allow utf-8 characters in the tex document
    \usepackage{fancyvrb} % verbatim replacement that allows latex
    \usepackage{grffile} % extends the file name processing of package graphics 
                         % to support a larger range 
    % The hyperref package gives us a pdf with properly built
    % internal navigation ('pdf bookmarks' for the table of contents,
    % internal cross-reference links, web links for URLs, etc.)
    \usepackage{hyperref}
    \usepackage{longtable} % longtable support required by pandoc >1.10
    \usepackage{booktabs}  % table support for pandoc > 1.12.2
    \usepackage[inline]{enumitem} % IRkernel/repr support (it uses the enumerate* environment)
    \usepackage[normalem]{ulem} % ulem is needed to support strikethroughs (\sout)
                                % normalem makes italics be italics, not underlines
    

    
    
    % Colors for the hyperref package
    \definecolor{urlcolor}{rgb}{0,.145,.698}
    \definecolor{linkcolor}{rgb}{.71,0.21,0.01}
    \definecolor{citecolor}{rgb}{.12,.54,.11}

    % ANSI colors
    \definecolor{ansi-black}{HTML}{3E424D}
    \definecolor{ansi-black-intense}{HTML}{282C36}
    \definecolor{ansi-red}{HTML}{E75C58}
    \definecolor{ansi-red-intense}{HTML}{B22B31}
    \definecolor{ansi-green}{HTML}{00A250}
    \definecolor{ansi-green-intense}{HTML}{007427}
    \definecolor{ansi-yellow}{HTML}{DDB62B}
    \definecolor{ansi-yellow-intense}{HTML}{B27D12}
    \definecolor{ansi-blue}{HTML}{208FFB}
    \definecolor{ansi-blue-intense}{HTML}{0065CA}
    \definecolor{ansi-magenta}{HTML}{D160C4}
    \definecolor{ansi-magenta-intense}{HTML}{A03196}
    \definecolor{ansi-cyan}{HTML}{60C6C8}
    \definecolor{ansi-cyan-intense}{HTML}{258F8F}
    \definecolor{ansi-white}{HTML}{C5C1B4}
    \definecolor{ansi-white-intense}{HTML}{A1A6B2}

    % commands and environments needed by pandoc snippets
    % extracted from the output of `pandoc -s`
    \providecommand{\tightlist}{%
      \setlength{\itemsep}{0pt}\setlength{\parskip}{0pt}}
    \DefineVerbatimEnvironment{Highlighting}{Verbatim}{commandchars=\\\{\}}
    % Add ',fontsize=\small' for more characters per line
    \newenvironment{Shaded}{}{}
    \newcommand{\KeywordTok}[1]{\textcolor[rgb]{0.00,0.44,0.13}{\textbf{{#1}}}}
    \newcommand{\DataTypeTok}[1]{\textcolor[rgb]{0.56,0.13,0.00}{{#1}}}
    \newcommand{\DecValTok}[1]{\textcolor[rgb]{0.25,0.63,0.44}{{#1}}}
    \newcommand{\BaseNTok}[1]{\textcolor[rgb]{0.25,0.63,0.44}{{#1}}}
    \newcommand{\FloatTok}[1]{\textcolor[rgb]{0.25,0.63,0.44}{{#1}}}
    \newcommand{\CharTok}[1]{\textcolor[rgb]{0.25,0.44,0.63}{{#1}}}
    \newcommand{\StringTok}[1]{\textcolor[rgb]{0.25,0.44,0.63}{{#1}}}
    \newcommand{\CommentTok}[1]{\textcolor[rgb]{0.38,0.63,0.69}{\textit{{#1}}}}
    \newcommand{\OtherTok}[1]{\textcolor[rgb]{0.00,0.44,0.13}{{#1}}}
    \newcommand{\AlertTok}[1]{\textcolor[rgb]{1.00,0.00,0.00}{\textbf{{#1}}}}
    \newcommand{\FunctionTok}[1]{\textcolor[rgb]{0.02,0.16,0.49}{{#1}}}
    \newcommand{\RegionMarkerTok}[1]{{#1}}
    \newcommand{\ErrorTok}[1]{\textcolor[rgb]{1.00,0.00,0.00}{\textbf{{#1}}}}
    \newcommand{\NormalTok}[1]{{#1}}
    
    % Additional commands for more recent versions of Pandoc
    \newcommand{\ConstantTok}[1]{\textcolor[rgb]{0.53,0.00,0.00}{{#1}}}
    \newcommand{\SpecialCharTok}[1]{\textcolor[rgb]{0.25,0.44,0.63}{{#1}}}
    \newcommand{\VerbatimStringTok}[1]{\textcolor[rgb]{0.25,0.44,0.63}{{#1}}}
    \newcommand{\SpecialStringTok}[1]{\textcolor[rgb]{0.73,0.40,0.53}{{#1}}}
    \newcommand{\ImportTok}[1]{{#1}}
    \newcommand{\DocumentationTok}[1]{\textcolor[rgb]{0.73,0.13,0.13}{\textit{{#1}}}}
    \newcommand{\AnnotationTok}[1]{\textcolor[rgb]{0.38,0.63,0.69}{\textbf{\textit{{#1}}}}}
    \newcommand{\CommentVarTok}[1]{\textcolor[rgb]{0.38,0.63,0.69}{\textbf{\textit{{#1}}}}}
    \newcommand{\VariableTok}[1]{\textcolor[rgb]{0.10,0.09,0.49}{{#1}}}
    \newcommand{\ControlFlowTok}[1]{\textcolor[rgb]{0.00,0.44,0.13}{\textbf{{#1}}}}
    \newcommand{\OperatorTok}[1]{\textcolor[rgb]{0.40,0.40,0.40}{{#1}}}
    \newcommand{\BuiltInTok}[1]{{#1}}
    \newcommand{\ExtensionTok}[1]{{#1}}
    \newcommand{\PreprocessorTok}[1]{\textcolor[rgb]{0.74,0.48,0.00}{{#1}}}
    \newcommand{\AttributeTok}[1]{\textcolor[rgb]{0.49,0.56,0.16}{{#1}}}
    \newcommand{\InformationTok}[1]{\textcolor[rgb]{0.38,0.63,0.69}{\textbf{\textit{{#1}}}}}
    \newcommand{\WarningTok}[1]{\textcolor[rgb]{0.38,0.63,0.69}{\textbf{\textit{{#1}}}}}
    
    
    % Define a nice break command that doesn't care if a line doesn't already
    % exist.
    \def\br{\hspace*{\fill} \\* }
    % Math Jax compatability definitions
    \def\gt{>}
    \def\lt{<}
    % Document parameters
    \title{Compute MCMD course - Medical Radiation Physics - hw week 4 - Samuel Wiqvist}
    
    
    

    % Pygments definitions
    
\makeatletter
\def\PY@reset{\let\PY@it=\relax \let\PY@bf=\relax%
    \let\PY@ul=\relax \let\PY@tc=\relax%
    \let\PY@bc=\relax \let\PY@ff=\relax}
\def\PY@tok#1{\csname PY@tok@#1\endcsname}
\def\PY@toks#1+{\ifx\relax#1\empty\else%
    \PY@tok{#1}\expandafter\PY@toks\fi}
\def\PY@do#1{\PY@bc{\PY@tc{\PY@ul{%
    \PY@it{\PY@bf{\PY@ff{#1}}}}}}}
\def\PY#1#2{\PY@reset\PY@toks#1+\relax+\PY@do{#2}}

\expandafter\def\csname PY@tok@w\endcsname{\def\PY@tc##1{\textcolor[rgb]{0.73,0.73,0.73}{##1}}}
\expandafter\def\csname PY@tok@c\endcsname{\let\PY@it=\textit\def\PY@tc##1{\textcolor[rgb]{0.25,0.50,0.50}{##1}}}
\expandafter\def\csname PY@tok@cp\endcsname{\def\PY@tc##1{\textcolor[rgb]{0.74,0.48,0.00}{##1}}}
\expandafter\def\csname PY@tok@k\endcsname{\let\PY@bf=\textbf\def\PY@tc##1{\textcolor[rgb]{0.00,0.50,0.00}{##1}}}
\expandafter\def\csname PY@tok@kp\endcsname{\def\PY@tc##1{\textcolor[rgb]{0.00,0.50,0.00}{##1}}}
\expandafter\def\csname PY@tok@kt\endcsname{\def\PY@tc##1{\textcolor[rgb]{0.69,0.00,0.25}{##1}}}
\expandafter\def\csname PY@tok@o\endcsname{\def\PY@tc##1{\textcolor[rgb]{0.40,0.40,0.40}{##1}}}
\expandafter\def\csname PY@tok@ow\endcsname{\let\PY@bf=\textbf\def\PY@tc##1{\textcolor[rgb]{0.67,0.13,1.00}{##1}}}
\expandafter\def\csname PY@tok@nb\endcsname{\def\PY@tc##1{\textcolor[rgb]{0.00,0.50,0.00}{##1}}}
\expandafter\def\csname PY@tok@nf\endcsname{\def\PY@tc##1{\textcolor[rgb]{0.00,0.00,1.00}{##1}}}
\expandafter\def\csname PY@tok@nc\endcsname{\let\PY@bf=\textbf\def\PY@tc##1{\textcolor[rgb]{0.00,0.00,1.00}{##1}}}
\expandafter\def\csname PY@tok@nn\endcsname{\let\PY@bf=\textbf\def\PY@tc##1{\textcolor[rgb]{0.00,0.00,1.00}{##1}}}
\expandafter\def\csname PY@tok@ne\endcsname{\let\PY@bf=\textbf\def\PY@tc##1{\textcolor[rgb]{0.82,0.25,0.23}{##1}}}
\expandafter\def\csname PY@tok@nv\endcsname{\def\PY@tc##1{\textcolor[rgb]{0.10,0.09,0.49}{##1}}}
\expandafter\def\csname PY@tok@no\endcsname{\def\PY@tc##1{\textcolor[rgb]{0.53,0.00,0.00}{##1}}}
\expandafter\def\csname PY@tok@nl\endcsname{\def\PY@tc##1{\textcolor[rgb]{0.63,0.63,0.00}{##1}}}
\expandafter\def\csname PY@tok@ni\endcsname{\let\PY@bf=\textbf\def\PY@tc##1{\textcolor[rgb]{0.60,0.60,0.60}{##1}}}
\expandafter\def\csname PY@tok@na\endcsname{\def\PY@tc##1{\textcolor[rgb]{0.49,0.56,0.16}{##1}}}
\expandafter\def\csname PY@tok@nt\endcsname{\let\PY@bf=\textbf\def\PY@tc##1{\textcolor[rgb]{0.00,0.50,0.00}{##1}}}
\expandafter\def\csname PY@tok@nd\endcsname{\def\PY@tc##1{\textcolor[rgb]{0.67,0.13,1.00}{##1}}}
\expandafter\def\csname PY@tok@s\endcsname{\def\PY@tc##1{\textcolor[rgb]{0.73,0.13,0.13}{##1}}}
\expandafter\def\csname PY@tok@sd\endcsname{\let\PY@it=\textit\def\PY@tc##1{\textcolor[rgb]{0.73,0.13,0.13}{##1}}}
\expandafter\def\csname PY@tok@si\endcsname{\let\PY@bf=\textbf\def\PY@tc##1{\textcolor[rgb]{0.73,0.40,0.53}{##1}}}
\expandafter\def\csname PY@tok@se\endcsname{\let\PY@bf=\textbf\def\PY@tc##1{\textcolor[rgb]{0.73,0.40,0.13}{##1}}}
\expandafter\def\csname PY@tok@sr\endcsname{\def\PY@tc##1{\textcolor[rgb]{0.73,0.40,0.53}{##1}}}
\expandafter\def\csname PY@tok@ss\endcsname{\def\PY@tc##1{\textcolor[rgb]{0.10,0.09,0.49}{##1}}}
\expandafter\def\csname PY@tok@sx\endcsname{\def\PY@tc##1{\textcolor[rgb]{0.00,0.50,0.00}{##1}}}
\expandafter\def\csname PY@tok@m\endcsname{\def\PY@tc##1{\textcolor[rgb]{0.40,0.40,0.40}{##1}}}
\expandafter\def\csname PY@tok@gh\endcsname{\let\PY@bf=\textbf\def\PY@tc##1{\textcolor[rgb]{0.00,0.00,0.50}{##1}}}
\expandafter\def\csname PY@tok@gu\endcsname{\let\PY@bf=\textbf\def\PY@tc##1{\textcolor[rgb]{0.50,0.00,0.50}{##1}}}
\expandafter\def\csname PY@tok@gd\endcsname{\def\PY@tc##1{\textcolor[rgb]{0.63,0.00,0.00}{##1}}}
\expandafter\def\csname PY@tok@gi\endcsname{\def\PY@tc##1{\textcolor[rgb]{0.00,0.63,0.00}{##1}}}
\expandafter\def\csname PY@tok@gr\endcsname{\def\PY@tc##1{\textcolor[rgb]{1.00,0.00,0.00}{##1}}}
\expandafter\def\csname PY@tok@ge\endcsname{\let\PY@it=\textit}
\expandafter\def\csname PY@tok@gs\endcsname{\let\PY@bf=\textbf}
\expandafter\def\csname PY@tok@gp\endcsname{\let\PY@bf=\textbf\def\PY@tc##1{\textcolor[rgb]{0.00,0.00,0.50}{##1}}}
\expandafter\def\csname PY@tok@go\endcsname{\def\PY@tc##1{\textcolor[rgb]{0.53,0.53,0.53}{##1}}}
\expandafter\def\csname PY@tok@gt\endcsname{\def\PY@tc##1{\textcolor[rgb]{0.00,0.27,0.87}{##1}}}
\expandafter\def\csname PY@tok@err\endcsname{\def\PY@bc##1{\setlength{\fboxsep}{0pt}\fcolorbox[rgb]{1.00,0.00,0.00}{1,1,1}{\strut ##1}}}
\expandafter\def\csname PY@tok@kc\endcsname{\let\PY@bf=\textbf\def\PY@tc##1{\textcolor[rgb]{0.00,0.50,0.00}{##1}}}
\expandafter\def\csname PY@tok@kd\endcsname{\let\PY@bf=\textbf\def\PY@tc##1{\textcolor[rgb]{0.00,0.50,0.00}{##1}}}
\expandafter\def\csname PY@tok@kn\endcsname{\let\PY@bf=\textbf\def\PY@tc##1{\textcolor[rgb]{0.00,0.50,0.00}{##1}}}
\expandafter\def\csname PY@tok@kr\endcsname{\let\PY@bf=\textbf\def\PY@tc##1{\textcolor[rgb]{0.00,0.50,0.00}{##1}}}
\expandafter\def\csname PY@tok@bp\endcsname{\def\PY@tc##1{\textcolor[rgb]{0.00,0.50,0.00}{##1}}}
\expandafter\def\csname PY@tok@fm\endcsname{\def\PY@tc##1{\textcolor[rgb]{0.00,0.00,1.00}{##1}}}
\expandafter\def\csname PY@tok@vc\endcsname{\def\PY@tc##1{\textcolor[rgb]{0.10,0.09,0.49}{##1}}}
\expandafter\def\csname PY@tok@vg\endcsname{\def\PY@tc##1{\textcolor[rgb]{0.10,0.09,0.49}{##1}}}
\expandafter\def\csname PY@tok@vi\endcsname{\def\PY@tc##1{\textcolor[rgb]{0.10,0.09,0.49}{##1}}}
\expandafter\def\csname PY@tok@vm\endcsname{\def\PY@tc##1{\textcolor[rgb]{0.10,0.09,0.49}{##1}}}
\expandafter\def\csname PY@tok@sa\endcsname{\def\PY@tc##1{\textcolor[rgb]{0.73,0.13,0.13}{##1}}}
\expandafter\def\csname PY@tok@sb\endcsname{\def\PY@tc##1{\textcolor[rgb]{0.73,0.13,0.13}{##1}}}
\expandafter\def\csname PY@tok@sc\endcsname{\def\PY@tc##1{\textcolor[rgb]{0.73,0.13,0.13}{##1}}}
\expandafter\def\csname PY@tok@dl\endcsname{\def\PY@tc##1{\textcolor[rgb]{0.73,0.13,0.13}{##1}}}
\expandafter\def\csname PY@tok@s2\endcsname{\def\PY@tc##1{\textcolor[rgb]{0.73,0.13,0.13}{##1}}}
\expandafter\def\csname PY@tok@sh\endcsname{\def\PY@tc##1{\textcolor[rgb]{0.73,0.13,0.13}{##1}}}
\expandafter\def\csname PY@tok@s1\endcsname{\def\PY@tc##1{\textcolor[rgb]{0.73,0.13,0.13}{##1}}}
\expandafter\def\csname PY@tok@mb\endcsname{\def\PY@tc##1{\textcolor[rgb]{0.40,0.40,0.40}{##1}}}
\expandafter\def\csname PY@tok@mf\endcsname{\def\PY@tc##1{\textcolor[rgb]{0.40,0.40,0.40}{##1}}}
\expandafter\def\csname PY@tok@mh\endcsname{\def\PY@tc##1{\textcolor[rgb]{0.40,0.40,0.40}{##1}}}
\expandafter\def\csname PY@tok@mi\endcsname{\def\PY@tc##1{\textcolor[rgb]{0.40,0.40,0.40}{##1}}}
\expandafter\def\csname PY@tok@il\endcsname{\def\PY@tc##1{\textcolor[rgb]{0.40,0.40,0.40}{##1}}}
\expandafter\def\csname PY@tok@mo\endcsname{\def\PY@tc##1{\textcolor[rgb]{0.40,0.40,0.40}{##1}}}
\expandafter\def\csname PY@tok@ch\endcsname{\let\PY@it=\textit\def\PY@tc##1{\textcolor[rgb]{0.25,0.50,0.50}{##1}}}
\expandafter\def\csname PY@tok@cm\endcsname{\let\PY@it=\textit\def\PY@tc##1{\textcolor[rgb]{0.25,0.50,0.50}{##1}}}
\expandafter\def\csname PY@tok@cpf\endcsname{\let\PY@it=\textit\def\PY@tc##1{\textcolor[rgb]{0.25,0.50,0.50}{##1}}}
\expandafter\def\csname PY@tok@c1\endcsname{\let\PY@it=\textit\def\PY@tc##1{\textcolor[rgb]{0.25,0.50,0.50}{##1}}}
\expandafter\def\csname PY@tok@cs\endcsname{\let\PY@it=\textit\def\PY@tc##1{\textcolor[rgb]{0.25,0.50,0.50}{##1}}}

\def\PYZbs{\char`\\}
\def\PYZus{\char`\_}
\def\PYZob{\char`\{}
\def\PYZcb{\char`\}}
\def\PYZca{\char`\^}
\def\PYZam{\char`\&}
\def\PYZlt{\char`\<}
\def\PYZgt{\char`\>}
\def\PYZsh{\char`\#}
\def\PYZpc{\char`\%}
\def\PYZdl{\char`\$}
\def\PYZhy{\char`\-}
\def\PYZsq{\char`\'}
\def\PYZdq{\char`\"}
\def\PYZti{\char`\~}
% for compatibility with earlier versions
\def\PYZat{@}
\def\PYZlb{[}
\def\PYZrb{]}
\makeatother


    % Exact colors from NB
    \definecolor{incolor}{rgb}{0.0, 0.0, 0.5}
    \definecolor{outcolor}{rgb}{0.545, 0.0, 0.0}



    
    % Prevent overflowing lines due to hard-to-break entities
    \sloppy 
    % Setup hyperref package
    \hypersetup{
      breaklinks=true,  % so long urls are correctly broken across lines
      colorlinks=true,
      urlcolor=urlcolor,
      linkcolor=linkcolor,
      citecolor=citecolor,
      }
    % Slightly bigger margins than the latex defaults
    
    \geometry{verbose,tmargin=1in,bmargin=1in,lmargin=1in,rmargin=1in}
    
    

    \begin{document}
    
    
    \maketitle
    
    

    
    \textbf{COMPUT MCMD course sping 2019, homework for week 4, Medical
Radiation Physics}

\textbf{By: Samuel Wiqvist}

    \paragraph{Collaborations with other
students:}\label{collaborations-with-other-students}

I have discussed the tasks with Maria Juhlin, but the work presented my
own work and understanding of the homework.

    \paragraph{Introduction:}\label{introduction}

We will I this task first sample photon path-lengths for a few different
materials; then we will sample a Compton scattering of a photon.

Setup, load packages that we will use:

    \begin{Verbatim}[commandchars=\\\{\}]
{\color{incolor}In [{\color{incolor}2}]:} \PY{k}{using} \PY{n}{PyPlot} \PY{c}{\PYZsh{} for plotting}
        \PY{k}{using} \PY{n}{KernelDensity} \PY{c}{\PYZsh{} to compute kernel density estimations}
        \PY{k}{using} \PY{n}{LaTeXStrings} \PY{c}{\PYZsh{} printing strings with LaTeX style}
\end{Verbatim}


    \begin{Verbatim}[commandchars=\\\{\}]
┌ Info: Recompiling stale cache file /home/samuel/.julia/compiled/v1.0/KernelDensity/4QyGx.ji for KernelDensity [5ab0869b-81aa-558d-bb23-cbf5423bbe9b]
└ @ Base loading.jl:1184

    \end{Verbatim}

    \paragraph{Sample photon
path-lengths:}\label{sample-photon-path-lengths}

We first load the data with different attenuation coefficient \(\mu\)
for the materials: water (H2O), aluminum (Al), iodine (I), and lead
(Pb). (The data is collected from:
https://physics.nist.gov/PhysRefData/Xcom/html/xcom1.html)

    \begin{Verbatim}[commandchars=\\\{\}]
{\color{incolor}In [{\color{incolor}8}]:} \PY{c}{\PYZsh{} load data}
        \PY{n}{include}\PY{p}{(}\PY{n}{pwd}\PY{p}{(}\PY{p}{)}\PY{o}{*}\PY{l+s}{\PYZdq{}}\PY{l+s}{/}\PY{l+s}{d}\PY{l+s}{a}\PY{l+s}{t}\PY{l+s}{a}\PY{l+s}{.}\PY{l+s}{j}\PY{l+s}{l}\PY{l+s}{\PYZdq{}}\PY{p}{)}\PY{p}{;}
\end{Verbatim}


    We now define the function \emph{generate\_photon\_path\_lengths} the
inputs to this function are the attenuation coefficient \(\mu\), and the
number of samples, the output of the function is the sampled
path-lengths.

    \begin{Verbatim}[commandchars=\\\{\}]
{\color{incolor}In [{\color{incolor}36}]:} \PY{c}{\PYZsh{} Function to sample N photon path lengths for some attenuation coefficient μ.}
         \PY{k}{function} \PY{n}{generate\PYZus{}photon\PYZus{}path\PYZus{}lengths}\PY{p}{(}\PY{n}{μ}\PY{o}{::}\PY{k+kt}{Real}\PY{p}{,} \PY{n}{N}\PY{o}{::}\PY{k+kt}{Int}\PY{p}{)}
             \PY{n}{d} \PY{o}{=} \PY{n}{zeros}\PY{p}{(}\PY{n}{N}\PY{p}{)}
             \PY{k}{for} \PY{n}{i} \PY{k+kp}{in} \PY{l+m+mi}{1}\PY{o}{:}\PY{n}{N}\PY{p}{;} \PY{n}{d}\PY{p}{[}\PY{n}{i}\PY{p}{]} \PY{o}{=} \PY{o}{\PYZhy{}}\PY{l+m+mi}{1}\PY{o}{/}\PY{n}{μ}\PY{o}{*}\PY{n}{log}\PY{p}{(}\PY{n}{rand}\PY{p}{(}\PY{p}{)}\PY{p}{)}\PY{p}{;} \PY{k}{end}
             \PY{k}{return} \PY{n}{d}
         \PY{k}{end}\PY{p}{;}
\end{Verbatim}


    We sample 1000 path-lengths for different energy-levels (i.e., different
attenuation coefficients \(\mu\)'s) for the materials: water (H2O),
aluminum (Al), iodine (I), and lead (Pb). The plots below show the
kernel density estimations of the distributions of the sampled
path-lengths. We conclude that the distributions for the path-lengths
vary for the different materials and that the distributions of
path-lengths for the water and aluminum are quite similar, while the
distributions of path-lengths for iodine and lead have other
characteristics. In particular, we see that we distributions of
path-lengths for iodine and lead are much more compressed at small
path-lengths for low energy-levels.

    \begin{Verbatim}[commandchars=\\\{\}]
{\color{incolor}In [{\color{incolor}39}]:} \PY{n}{PyPlot}\PY{o}{.}\PY{n}{figure}\PY{p}{(}\PY{n}{figsize}\PY{o}{=}\PY{p}{(}\PY{l+m+mi}{5}\PY{p}{,}\PY{l+m+mi}{5}\PY{p}{)}\PY{p}{)}
         \PY{k}{for} \PY{n}{i} \PY{o}{=} \PY{l+m+mi}{1}\PY{o}{:}\PY{n}{size}\PY{p}{(}\PY{n}{data\PYZus{}h2o}\PY{p}{,}\PY{l+m+mi}{1}\PY{p}{)}
             \PY{n}{μ} \PY{o}{=} \PY{n}{sum}\PY{p}{(}\PY{n}{data\PYZus{}h2o}\PY{p}{[}\PY{n}{i}\PY{p}{,}\PY{k}{end}\PY{o}{\PYZhy{}}\PY{l+m+mi}{1}\PY{p}{]}\PY{p}{)}
             \PY{n}{N} \PY{o}{=} \PY{l+m+mi}{10}\PY{o}{\PYZca{}}\PY{l+m+mi}{3}
             \PY{n}{d} \PY{o}{=} \PY{n}{generate\PYZus{}photon\PYZus{}path\PYZus{}lengths}\PY{p}{(}\PY{n}{μ}\PY{p}{,} \PY{n}{N}\PY{p}{)}
             \PY{n}{kde\PYZus{}est} \PY{o}{=} \PY{n}{kde}\PY{p}{(}\PY{n}{d}\PY{p}{)}
             \PY{n}{PyPlot}\PY{o}{.}\PY{n}{plot}\PY{p}{(}\PY{n}{kde\PYZus{}est}\PY{o}{.}\PY{n}{x}\PY{p}{,} \PY{n}{kde\PYZus{}est}\PY{o}{.}\PY{n}{density}\PY{p}{,} \PY{n}{label}\PY{o}{=}\PY{n}{string}\PY{p}{(}\PY{n}{data\PYZus{}h2o}\PY{p}{[}\PY{n}{i}\PY{p}{,}\PY{l+m+mi}{1}\PY{p}{]}\PY{p}{)}\PY{o}{*}\PY{l+s}{\PYZdq{}}\PY{l+s}{M}\PY{l+s}{e}\PY{l+s}{V}\PY{l+s}{\PYZdq{}}\PY{p}{)}
             \PY{n}{PyPlot}\PY{o}{.}\PY{n}{title}\PY{p}{(}\PY{l+s}{\PYZdq{}}\PY{l+s}{H}\PY{l+s}{2}\PY{l+s}{O}\PY{l+s}{\PYZdq{}}\PY{p}{)}
         \PY{k}{end}
         \PY{n}{PyPlot}\PY{o}{.}\PY{n}{legend}\PY{p}{(}\PY{p}{)}
         
         \PY{n}{PyPlot}\PY{o}{.}\PY{n}{figure}\PY{p}{(}\PY{n}{figsize}\PY{o}{=}\PY{p}{(}\PY{l+m+mi}{5}\PY{p}{,}\PY{l+m+mi}{5}\PY{p}{)}\PY{p}{)}
         \PY{k}{for} \PY{n}{i} \PY{o}{=} \PY{l+m+mi}{1}\PY{o}{:}\PY{n}{size}\PY{p}{(}\PY{n}{data\PYZus{}h2o}\PY{p}{,}\PY{l+m+mi}{1}\PY{p}{)}
             \PY{n}{μ} \PY{o}{=} \PY{n}{sum}\PY{p}{(}\PY{n}{data\PYZus{}Al}\PY{p}{[}\PY{n}{i}\PY{p}{,}\PY{k}{end}\PY{o}{\PYZhy{}}\PY{l+m+mi}{1}\PY{p}{]}\PY{p}{)}
             \PY{n}{N} \PY{o}{=} \PY{l+m+mi}{10}\PY{o}{\PYZca{}}\PY{l+m+mi}{3}
             \PY{n}{d} \PY{o}{=} \PY{n}{generate\PYZus{}photon\PYZus{}path\PYZus{}lengths}\PY{p}{(}\PY{n}{μ}\PY{p}{,} \PY{n}{N}\PY{p}{)}
             \PY{n}{kde\PYZus{}est} \PY{o}{=} \PY{n}{kde}\PY{p}{(}\PY{n}{d}\PY{p}{)}
             \PY{n}{PyPlot}\PY{o}{.}\PY{n}{plot}\PY{p}{(}\PY{n}{kde\PYZus{}est}\PY{o}{.}\PY{n}{x}\PY{p}{,} \PY{n}{kde\PYZus{}est}\PY{o}{.}\PY{n}{density}\PY{p}{,} \PY{n}{label}\PY{o}{=}\PY{n}{string}\PY{p}{(}\PY{n}{data\PYZus{}Al}\PY{p}{[}\PY{n}{i}\PY{p}{,}\PY{l+m+mi}{1}\PY{p}{]}\PY{p}{)}\PY{o}{*}\PY{l+s}{\PYZdq{}}\PY{l+s}{M}\PY{l+s}{e}\PY{l+s}{V}\PY{l+s}{\PYZdq{}}\PY{p}{)}
             \PY{n}{PyPlot}\PY{o}{.}\PY{n}{title}\PY{p}{(}\PY{l+s}{\PYZdq{}}\PY{l+s}{A}\PY{l+s}{l}\PY{l+s}{\PYZdq{}}\PY{p}{)}
         \PY{k}{end}
         \PY{n}{PyPlot}\PY{o}{.}\PY{n}{legend}\PY{p}{(}\PY{p}{)}
         
         \PY{n}{PyPlot}\PY{o}{.}\PY{n}{figure}\PY{p}{(}\PY{n}{figsize}\PY{o}{=}\PY{p}{(}\PY{l+m+mi}{5}\PY{p}{,}\PY{l+m+mi}{5}\PY{p}{)}\PY{p}{)}
         \PY{k}{for} \PY{n}{i} \PY{o}{=} \PY{l+m+mi}{1}\PY{o}{:}\PY{n}{size}\PY{p}{(}\PY{n}{data\PYZus{}h2o}\PY{p}{,}\PY{l+m+mi}{1}\PY{p}{)}
             \PY{n}{μ} \PY{o}{=} \PY{n}{sum}\PY{p}{(}\PY{n}{data\PYZus{}I}\PY{p}{[}\PY{n}{i}\PY{p}{,}\PY{k}{end}\PY{o}{\PYZhy{}}\PY{l+m+mi}{1}\PY{p}{]}\PY{p}{)}
             \PY{n}{N} \PY{o}{=} \PY{l+m+mi}{10}\PY{o}{\PYZca{}}\PY{l+m+mi}{3}
             \PY{n}{d} \PY{o}{=} \PY{n}{generate\PYZus{}photon\PYZus{}path\PYZus{}lengths}\PY{p}{(}\PY{n}{μ}\PY{p}{,} \PY{n}{N}\PY{p}{)}
             \PY{n}{kde\PYZus{}est} \PY{o}{=} \PY{n}{kde}\PY{p}{(}\PY{n}{d}\PY{p}{)}
             \PY{n}{PyPlot}\PY{o}{.}\PY{n}{plot}\PY{p}{(}\PY{n}{kde\PYZus{}est}\PY{o}{.}\PY{n}{x}\PY{p}{,} \PY{n}{kde\PYZus{}est}\PY{o}{.}\PY{n}{density}\PY{p}{,} \PY{n}{label}\PY{o}{=}\PY{n}{string}\PY{p}{(}\PY{n}{data\PYZus{}I}\PY{p}{[}\PY{n}{i}\PY{p}{,}\PY{l+m+mi}{1}\PY{p}{]}\PY{p}{)}\PY{o}{*}\PY{l+s}{\PYZdq{}}\PY{l+s}{M}\PY{l+s}{e}\PY{l+s}{V}\PY{l+s}{\PYZdq{}}\PY{p}{)}
             \PY{n}{PyPlot}\PY{o}{.}\PY{n}{title}\PY{p}{(}\PY{l+s}{\PYZdq{}}\PY{l+s}{I}\PY{l+s}{\PYZdq{}}\PY{p}{)}
         \PY{k}{end}
         \PY{n}{PyPlot}\PY{o}{.}\PY{n}{legend}\PY{p}{(}\PY{p}{)}
         
         \PY{n}{PyPlot}\PY{o}{.}\PY{n}{figure}\PY{p}{(}\PY{n}{figsize}\PY{o}{=}\PY{p}{(}\PY{l+m+mi}{5}\PY{p}{,}\PY{l+m+mi}{5}\PY{p}{)}\PY{p}{)}
         \PY{k}{for} \PY{n}{i} \PY{o}{=} \PY{l+m+mi}{1}\PY{o}{:}\PY{n}{size}\PY{p}{(}\PY{n}{data\PYZus{}h2o}\PY{p}{,}\PY{l+m+mi}{1}\PY{p}{)}
             \PY{n}{μ} \PY{o}{=} \PY{n}{sum}\PY{p}{(}\PY{n}{data\PYZus{}Pb}\PY{p}{[}\PY{n}{i}\PY{p}{,}\PY{k}{end}\PY{o}{\PYZhy{}}\PY{l+m+mi}{1}\PY{p}{]}\PY{p}{)}
             \PY{n}{N} \PY{o}{=} \PY{l+m+mi}{10}\PY{o}{\PYZca{}}\PY{l+m+mi}{3}
             \PY{n}{d} \PY{o}{=} \PY{n}{generate\PYZus{}photon\PYZus{}path\PYZus{}lengths}\PY{p}{(}\PY{n}{μ}\PY{p}{,} \PY{n}{N}\PY{p}{)}
             \PY{n}{kde\PYZus{}est} \PY{o}{=} \PY{n}{kde}\PY{p}{(}\PY{n}{d}\PY{p}{)}
             \PY{n}{PyPlot}\PY{o}{.}\PY{n}{plot}\PY{p}{(}\PY{n}{kde\PYZus{}est}\PY{o}{.}\PY{n}{x}\PY{p}{,} \PY{n}{kde\PYZus{}est}\PY{o}{.}\PY{n}{density}\PY{p}{,} \PY{n}{label}\PY{o}{=}\PY{n}{string}\PY{p}{(}\PY{n}{data\PYZus{}Pb}\PY{p}{[}\PY{n}{i}\PY{p}{,}\PY{l+m+mi}{1}\PY{p}{]}\PY{p}{)}\PY{o}{*}\PY{l+s}{\PYZdq{}}\PY{l+s}{M}\PY{l+s}{e}\PY{l+s}{V}\PY{l+s}{\PYZdq{}}\PY{p}{)}
             \PY{n}{PyPlot}\PY{o}{.}\PY{n}{title}\PY{p}{(}\PY{l+s}{\PYZdq{}}\PY{l+s}{P}\PY{l+s}{b}\PY{l+s}{\PYZdq{}}\PY{p}{)}
         \PY{k}{end}
         \PY{n}{PyPlot}\PY{o}{.}\PY{n}{legend}\PY{p}{(}\PY{p}{)}\PY{p}{;}
\end{Verbatim}


    \begin{center}
    \adjustimage{max size={0.9\linewidth}{0.9\paperheight}}{output_9_0.png}
    \end{center}
    { \hspace*{\fill} \\}
    
    \begin{center}
    \adjustimage{max size={0.9\linewidth}{0.9\paperheight}}{output_9_1.png}
    \end{center}
    { \hspace*{\fill} \\}
    
    \begin{center}
    \adjustimage{max size={0.9\linewidth}{0.9\paperheight}}{output_9_2.png}
    \end{center}
    { \hspace*{\fill} \\}
    
    \begin{center}
    \adjustimage{max size={0.9\linewidth}{0.9\paperheight}}{output_9_3.png}
    \end{center}
    { \hspace*{\fill} \\}
    
    \paragraph{Sample Compton scattering:}\label{sample-compton-scattering}

The next task is to sample a Compton scattering of a photon.

Firstly, we define the function \emph{compton\_scattering} that takes as
input the initial photon energy \(hv\) and returns the samples new angle
\(\theta\) and the new energy \(hv'\).

    \begin{Verbatim}[commandchars=\\\{\}]
{\color{incolor}In [{\color{incolor}37}]:} \PY{c}{\PYZsh{} Function to generate one compton scattering event.}
         \PY{k}{function} \PY{n}{compton\PYZus{}scattering}\PY{p}{(}\PY{n}{hv}\PY{o}{::}\PY{k+kt}{Float64}\PY{p}{)}
         
             \PY{n}{h} \PY{o}{=} \PY{l+m+mi}{1}\PY{p}{;} \PY{n}{m\PYZus{}0} \PY{o}{=} \PY{l+m+mi}{1}\PY{p}{;} \PY{n}{c} \PY{o}{=} \PY{l+m+mi}{1}
         
             \PY{n}{λ} \PY{o}{=} \PY{n}{hv} \PY{o}{/} \PY{p}{(}\PY{n}{m\PYZus{}0}\PY{o}{*}\PY{n}{c}\PY{o}{\PYZca{}}\PY{l+m+mi}{2}\PY{p}{)}
             \PY{n}{Q} \PY{o}{=} \PY{p}{(}\PY{l+m+mi}{2}\PY{o}{*}\PY{n}{λ} \PY{o}{+} \PY{l+m+mi}{1}\PY{p}{)}\PY{o}{/}\PY{p}{(}\PY{l+m+mi}{2}\PY{o}{*}\PY{n}{λ} \PY{o}{+} \PY{l+m+mi}{9}\PY{p}{)}
         
             \PY{n}{run\PYZus{}sampler} \PY{o}{=} \PY{k+kc}{true}
         
             \PY{n}{nbr\PYZus{}sampels} \PY{o}{=} \PY{l+m+mi}{0}\PY{p}{;} \PY{n}{hv\PYZus{}prime} \PY{o}{=} \PY{l+m+mf}{0.}\PY{p}{;} \PY{n}{θ} \PY{o}{=} \PY{l+m+mf}{0.}\PY{p}{;} \PY{n}{cosθ} \PY{o}{=} \PY{l+m+mf}{0.}
         
             \PY{k}{while} \PY{n}{run\PYZus{}sampler}
         
                 \PY{n}{nbr\PYZus{}sampels} \PY{o}{=} \PY{n}{nbr\PYZus{}sampels} \PY{o}{+} \PY{l+m+mi}{1}
         
                 \PY{n}{R1} \PY{o}{=} \PY{n}{rand}\PY{p}{(}\PY{p}{)}\PY{p}{;} \PY{n}{R2} \PY{o}{=} \PY{n}{rand}\PY{p}{(}\PY{p}{)}\PY{p}{;} \PY{n}{R3} \PY{o}{=} \PY{n}{rand}\PY{p}{(}\PY{p}{)}
         
                 \PY{k}{if} \PY{n}{R1} \PY{o}{\PYZlt{}} \PY{n}{Q}
                     \PY{n}{ρ} \PY{o}{=} \PY{l+m+mi}{1} \PY{o}{+} \PY{l+m+mi}{2}\PY{o}{*}\PY{n}{λ}\PY{o}{*}\PY{n}{R2}
                     \PY{k}{if} \PY{n}{R3} \PY{o}{\PYZgt{}} \PY{p}{(}\PY{l+m+mi}{4}\PY{o}{*}\PY{p}{(}\PY{n}{ρ}\PY{o}{\PYZhy{}}\PY{l+m+mi}{1}\PY{p}{)}\PY{p}{)}\PY{o}{/}\PY{n}{ρ}\PY{o}{\PYZca{}}\PY{l+m+mi}{2}
                         \PY{c}{\PYZsh{} generate new sample}
                     \PY{k}{else}
                         \PY{n}{cosθ} \PY{o}{=} \PY{l+m+mi}{1}\PY{o}{\PYZhy{}}\PY{l+m+mi}{2}\PY{o}{*}\PY{n}{R2}
                         \PY{n}{θ} \PY{o}{=} \PY{n}{acos}\PY{p}{(}\PY{n}{cosθ}\PY{p}{)}
                         \PY{n}{θ} \PY{o}{=} \PY{n}{θ}\PY{o}{*}\PY{l+m+mi}{180}\PY{o}{/}\PY{n+nb}{π}
                         \PY{n}{hv\PYZus{}prime} \PY{o}{=} \PY{n}{hv}\PY{o}{/}\PY{n}{ρ}
                         \PY{n}{run\PYZus{}sampler} \PY{o}{=} \PY{k+kc}{false}
                     \PY{k}{end}
                 \PY{k}{else}
                     \PY{n}{ρ} \PY{o}{=} \PY{p}{(}\PY{l+m+mi}{2}\PY{o}{*}\PY{n}{λ} \PY{o}{+} \PY{l+m+mi}{1}\PY{p}{)}\PY{o}{/}\PY{p}{(}\PY{l+m+mi}{2}\PY{o}{*}\PY{n}{λ}\PY{o}{*}\PY{n}{R2} \PY{o}{+} \PY{l+m+mi}{1}\PY{p}{)}
                     \PY{k}{if} \PY{n}{R3} \PY{o}{\PYZgt{}} \PY{p}{(}\PY{p}{(}\PY{l+m+mi}{1}\PY{o}{\PYZhy{}}\PY{p}{(}\PY{n}{ρ}\PY{o}{\PYZhy{}}\PY{l+m+mi}{1}\PY{p}{)}\PY{o}{/}\PY{n}{λ}\PY{p}{)}\PY{o}{\PYZca{}}\PY{l+m+mi}{2} \PY{o}{+} \PY{l+m+mi}{1}\PY{o}{/}\PY{n}{ρ}\PY{p}{)}\PY{o}{/}\PY{l+m+mi}{2}
                         \PY{c}{\PYZsh{} generate new sample}
                     \PY{k}{else}
                         \PY{n}{cosθ} \PY{o}{=} \PY{l+m+mi}{1}\PY{o}{\PYZhy{}}\PY{p}{(}\PY{n}{ρ}\PY{o}{\PYZhy{}}\PY{l+m+mi}{1}\PY{p}{)}\PY{o}{/}\PY{n}{λ}
                         \PY{n}{θ} \PY{o}{=} \PY{n}{acos}\PY{p}{(}\PY{n}{cosθ}\PY{p}{)}
                         \PY{n}{θ} \PY{o}{=} \PY{n}{θ}\PY{o}{*}\PY{l+m+mi}{180}\PY{o}{/}\PY{n+nb}{π}
                         \PY{n}{hv\PYZus{}prime} \PY{o}{=} \PY{n}{hv}\PY{o}{/}\PY{n}{ρ}
                         \PY{n}{run\PYZus{}sampler} \PY{o}{=} \PY{k+kc}{false}
                     \PY{k}{end}
                 \PY{k}{end}
         
             \PY{k}{end}
         
             \PY{k}{return} \PY{p}{[}\PY{n}{hv\PYZus{}prime}\PY{p}{;} \PY{n}{θ}\PY{p}{;} \PY{n}{cosθ}\PY{p}{;} \PY{n}{nbr\PYZus{}sampels}\PY{p}{]}
         
         \PY{k}{end}\PY{p}{;}
\end{Verbatim}


    For our simulations we sample \(100000\) Compton scatterings with the
initial energies
\(hv = [0.05, 0.06, 0.1, 0.2, 0.4, 0.6, 1.0] \, (MeV)\).

    \begin{Verbatim}[commandchars=\\\{\}]
{\color{incolor}In [{\color{incolor}18}]:} \PY{c}{\PYZsh{} sample compton scattering}
         \PY{n}{N} \PY{o}{=} \PY{l+m+mi}{100000}
         \PY{n}{energy\PYZus{}levels} \PY{o}{=} \PY{n}{data\PYZus{}h2o}\PY{p}{[}\PY{o}{:}\PY{p}{,}\PY{l+m+mi}{1}\PY{p}{]}
         \PY{n}{samples\PYZus{}compton\PYZus{}scattering} \PY{o}{=} \PY{n}{zeros}\PY{p}{(}\PY{l+m+mi}{4}\PY{p}{,}\PY{n}{N}\PY{p}{,}\PY{n}{length}\PY{p}{(}\PY{n}{energy\PYZus{}levels}\PY{p}{)}\PY{p}{)}
         
         \PY{k}{for} \PY{n}{i} \PY{k+kp}{in} \PY{l+m+mi}{1}\PY{o}{:}\PY{n}{length}\PY{p}{(}\PY{n}{energy\PYZus{}levels}\PY{p}{)}
             \PY{k}{for} \PY{n}{j} \PY{o}{=} \PY{l+m+mi}{1}\PY{o}{:}\PY{n}{N}
                 \PY{n}{samples\PYZus{}compton\PYZus{}scattering}\PY{p}{[}\PY{o}{:}\PY{p}{,}\PY{n}{j}\PY{p}{,}\PY{n}{i}\PY{p}{]} \PY{o}{=} \PY{n}{compton\PYZus{}scattering}\PY{p}{(}\PY{n}{energy\PYZus{}levels}\PY{p}{[}\PY{n}{i}\PY{p}{]}\PY{p}{)}
             \PY{k}{end}
         \PY{k}{end}
\end{Verbatim}


    Finally, we plot the histograms for new angels \(\theta\) and the new
energies \(hv'\) for the different initial energies. From the figures
below we conclude that the new energy \(hv'\) increases when we increase
the input energy \(hv\), we also conclude that peak angel is around
\(50^{\circ}\), and that the distribution of the angle \(\theta\) gets
more compressed when we increase the input energy \(hv\).

    \begin{Verbatim}[commandchars=\\\{\}]
{\color{incolor}In [{\color{incolor}34}]:} \PY{n}{PyPlot}\PY{o}{.}\PY{n}{figure}\PY{p}{(}\PY{n}{figsize}\PY{o}{=}\PY{p}{(}\PY{l+m+mi}{10}\PY{p}{,}\PY{l+m+mi}{40}\PY{p}{)}\PY{p}{)}
         
         \PY{n}{energy\PYZus{}level} \PY{o}{=} \PY{l+m+mi}{0}
         
         \PY{k}{for} \PY{n}{i} \PY{k+kp}{in} \PY{l+m+mi}{1}\PY{o}{:}\PY{l+m+mi}{2}\PY{o}{:}\PY{l+m+mi}{2}\PY{o}{*}\PY{n}{length}\PY{p}{(}\PY{n}{energy\PYZus{}levels}\PY{p}{)}
             \PY{n}{PyPlot}\PY{o}{.}\PY{n}{figure}\PY{p}{(}\PY{p}{)}
             \PY{k+kd}{global} \PY{n}{energy\PYZus{}level} \PY{o}{=} \PY{n}{energy\PYZus{}level} \PY{o}{+} \PY{l+m+mi}{1}
             \PY{n}{PyPlot}\PY{o}{.}\PY{n}{subplot}\PY{p}{(}\PY{l+m+mi}{1}\PY{p}{,}\PY{l+m+mi}{2}\PY{p}{,}\PY{l+m+mi}{1}\PY{p}{)}
             \PY{n}{h} \PY{o}{=} \PY{n}{PyPlot}\PY{o}{.}\PY{n}{plt}\PY{p}{[}\PY{o}{:}\PY{n}{hist}\PY{p}{]}\PY{p}{(}\PY{n}{samples\PYZus{}compton\PYZus{}scattering}\PY{p}{[}\PY{l+m+mi}{1}\PY{p}{,}\PY{o}{:}\PY{p}{,}\PY{n}{energy\PYZus{}level}\PY{p}{]}\PY{p}{,}\PY{l+m+mi}{100}\PY{p}{)}
             \PY{n}{PyPlot}\PY{o}{.}\PY{n}{ylabel}\PY{p}{(}\PY{l+s}{\PYZdq{}}\PY{l+s}{F}\PY{l+s}{r}\PY{l+s}{e}\PY{l+s}{q}\PY{l+s}{.}\PY{l+s}{\PYZdq{}}\PY{p}{)}
             \PY{n}{PyPlot}\PY{o}{.}\PY{n}{xlabel}\PY{p}{(}\PY{l+s}{\PYZdq{}}\PY{l+s}{h}\PY{l+s}{v}\PY{l+s}{\PYZsq{}}\PY{l+s}{\PYZdq{}}\PY{p}{)}
             \PY{n}{PyPlot}\PY{o}{.}\PY{n}{subplot}\PY{p}{(}\PY{l+m+mi}{1}\PY{p}{,}\PY{l+m+mi}{2}\PY{p}{,}\PY{l+m+mi}{2}\PY{p}{)}
             \PY{n}{h} \PY{o}{=} \PY{n}{PyPlot}\PY{o}{.}\PY{n}{plt}\PY{p}{[}\PY{o}{:}\PY{n}{hist}\PY{p}{]}\PY{p}{(}\PY{n}{samples\PYZus{}compton\PYZus{}scattering}\PY{p}{[}\PY{l+m+mi}{2}\PY{p}{,}\PY{o}{:}\PY{p}{,}\PY{n}{energy\PYZus{}level}\PY{p}{]}\PY{p}{,}\PY{l+m+mi}{100}\PY{p}{)}
             \PY{n}{PyPlot}\PY{o}{.}\PY{n}{xlabel}\PY{p}{(}\PY{n}{L}\PY{l+s}{\PYZdq{}}\PY{l+s+se}{\PYZbs{}t}\PY{l+s}{h}\PY{l+s}{e}\PY{l+s}{t}\PY{l+s}{a}\PY{l+s}{\PYZca{}}\PY{l+s}{\PYZob{}}\PY{l+s}{\PYZbs{}}\PY{l+s}{c}\PY{l+s}{i}\PY{l+s}{r}\PY{l+s}{c}\PY{l+s}{\PYZcb{}}\PY{l+s}{\PYZdq{}}\PY{p}{)}
             \PY{n}{PyPlot}\PY{o}{.}\PY{n}{suptitle}\PY{p}{(}\PY{l+s}{\PYZdq{}}\PY{l+s}{h}\PY{l+s}{v}\PY{l+s}{:}\PY{l+s}{\PYZdq{}}\PY{o}{*}\PY{n}{string}\PY{p}{(}\PY{n}{data\PYZus{}h2o}\PY{p}{[}\PY{n}{energy\PYZus{}level}\PY{p}{,}\PY{l+m+mi}{1}\PY{p}{]}\PY{p}{)}\PY{o}{*}\PY{l+s}{\PYZdq{}}\PY{l+s}{M}\PY{l+s}{e}\PY{l+s}{V}\PY{l+s}{\PYZdq{}}\PY{p}{)}
             
         \PY{k}{end}\PY{p}{;}
\end{Verbatim}


    \begin{center}
    \adjustimage{max size={0.9\linewidth}{0.9\paperheight}}{output_15_0.png}
    \end{center}
    { \hspace*{\fill} \\}
    
    \begin{center}
    \adjustimage{max size={0.9\linewidth}{0.9\paperheight}}{output_15_1.png}
    \end{center}
    { \hspace*{\fill} \\}
    
    \begin{center}
    \adjustimage{max size={0.9\linewidth}{0.9\paperheight}}{output_15_2.png}
    \end{center}
    { \hspace*{\fill} \\}
    
    \begin{center}
    \adjustimage{max size={0.9\linewidth}{0.9\paperheight}}{output_15_3.png}
    \end{center}
    { \hspace*{\fill} \\}
    
    \begin{center}
    \adjustimage{max size={0.9\linewidth}{0.9\paperheight}}{output_15_4.png}
    \end{center}
    { \hspace*{\fill} \\}
    
    \begin{center}
    \adjustimage{max size={0.9\linewidth}{0.9\paperheight}}{output_15_5.png}
    \end{center}
    { \hspace*{\fill} \\}
    
    \begin{center}
    \adjustimage{max size={0.9\linewidth}{0.9\paperheight}}{output_15_6.png}
    \end{center}
    { \hspace*{\fill} \\}
    
    \begin{Verbatim}[commandchars=\\\{\}]
┌ Warning: `getindex(o::PyObject, s::Symbol)` is deprecated in favor of dot overloading (`getproperty`) so elements should now be accessed as e.g. `o.s` instead of `o[:s]`.
│   caller = top-level scope at In[34]:10
└ @ Core ./In[34]:10
┌ Warning: `getindex(o::PyObject, s::Symbol)` is deprecated in favor of dot overloading (`getproperty`) so elements should now be accessed as e.g. `o.s` instead of `o[:s]`.
│   caller = top-level scope at In[34]:14
└ @ Core ./In[34]:14

    \end{Verbatim}


    % Add a bibliography block to the postdoc
    
    
    
    \end{document}
